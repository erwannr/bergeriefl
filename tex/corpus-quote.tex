% -----------------------------------------------------------------
% Copyright (C) 2025 by Le cercle floridien (Erwann Rogard)
% Source repository: https://github.com/erwannr/cerclefl
% 
% LaTeX code:
% Released under the LaTeX Project Public License v1.3c or later
% See http://www.latex-project.org/lppl.txt
% 
% Written content (text):
% CC BY-NC-SA 4.0
% -----------------------------------------------------------------

\DTLnewdb{corpus-quote}

\DTLnewrow{corpus-quote}
\DTLnewdbentry{corpus-quote}{id}{zola-lerouge}
\DTLnewdbentry{corpus-quote}{authorid}{zola}
\DTLnewdbentry{corpus-quote}{source}{\cite{ternois1937}}
\DTLnewdbentry{corpus-quote}{text}{Stendhal est un logicien qui part de la logique et qui arrive souvent à la vérité, en passant par dessus l'observation; dédain du corps, silence sur les éléments physiologiques de l'homme, et sur les rôle des milieux ambiants. Il met l'être humain à part dans la nature et déclare ensuite que l'âme seule est noble et a droit de cité dans la littérature.  Dans le petit drame muet où Julien se fait un devoir de prendre la main de Mme de Rênal nous pourrions être n'importe où, la scène resterait la même, pourvu qu'il fit noir. Donnez à Julien un champ de bataille digne de lui, il triomphera superbement, sans descendre à de continuelles roueries de diplomate. Il est donc bien l'enfant de cette heure historique, un garçon d'une intelligence supérieure obligé par tempéramentde faire une grande fortune, qui est venu trop tard pour être un des maréchaux de Napoléon, et qui se résout à passer par les sacristies et à opérer en valet hypocrite. Dès lors, son caractère s'éclaire, on comprend ses soumission et ses révoltes, ses tendresses et ses cruautés, ses tromperies et ses franchises; il montre d'ailleurs autant de naïveté que d'adresse.}

\DTLnewrow{corpus-quote}
\DTLnewdbentry{corpus-quote}{id}{lerouge-tous-ces-marchands}
\DTLnewdbentry{corpus-quote}{authorid}{stendhal}
\DTLnewdbentry{corpus-quote}{source}{\cite[p.~10]{matsumoto1997}}
\DTLnewdbentry{corpus-quote}{text}{Tous ces marchands de toile me portent envie; eh bien! j'aime assez qu'ils voient passer mes enfants sous la conduite de leur précepteur.}
