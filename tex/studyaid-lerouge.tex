% -----------------------------------------------------------------
% Copyright (C) 2025 by Le cercle floridien (Erwann Rogard)
% Source repository: https://github.com/erwannr/cerclefl
% 
% LaTeX code:
% Released under the LaTeX Project Public License v1.3c or later
% See http://www.latex-project.org/lppl.txt
% 
% Written content (text):
% CC BY-NC-SA 4.0
% -----------------------------------------------------------------

\DTLnewdb{studyaid-lerouge}

\DTLnewrow{studyaid-lerouge}
\DTLnewdbentry{studyaid-lerouge}{id}{lerouge-tous-ces-marchands}

\DTLnewdbentry{studyaid-lerouge}{question}{Comment le caractère de M. de Rênal se révèle-t-il dans ce chapitre ? Quels sont les divers mobiles qui le poussent à donner un précepteur à ses enfants ? Dans quel ordre présente-t-il ses arguments à sa femme?\cite{ternois1937}}

\DTLnewdbentry{studyaid-lerouge}{quote-id}{lerouge-tous-ces-marchands}

\DTLnewdbentry{studyaid-lerouge}{answer}{M. de Rênal recrute Sorel comme précepteur pour soutenir leur rang. En tant que réactionnaire, il rationalise ouvertement d'accepter chez lui le neveu d'un libéral (le vieux chirurgien-major), c'est à dire un ennemi idéologique. C'est, au figuré, un \emph{pharisien}\endnote{Lequel du sens figuré accepté de \emph{pharisien}\cite{larousse-pharisien}, et celui du fragment des pensées de Blaise Pascal\cite{pascal-fragment-469}, précède l'autre?}.}
